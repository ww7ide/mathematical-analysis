\chapter{收敛}

从本章开始,我们终于进入分析的领域。数学的这一分支在很大程度上是建立在 ``收敛'' 这一概念之上的。借助收敛,我们在某种意义上得以把无穷多个数(或向量)``加'' 在一起。能够处理这样的无限运算,正是分析与代数之间本质区别之所在。

试图把关于数列收敛的朴素想法加以公理化,会自然地引出 ``距离''、``点的邻域'' 以及 ``度量空间'' 等概念——这正是第 1 节的主题。在数列这一特殊情形中,我们可以利用数域 $\K$ 的向量空间结构,对这种情形下各种证明的分析表明:只要有某种类似 ``绝对值'' 的概念可用,其中大多数证明都可以推广到向量空间中的向量序列。于是我们很自然地被引向去定义 ``赋范向量空间''——度量空间中一个尤其重要的类别。

在所有赋范向量空间中,内积空间由于其结构更为丰富而显得格外重要,并且它们的几何性质与我们熟悉的平面 Euclidean 几何非常相似。事实上,在初等分析中,最重要的一类内积空间就是 $m$-维 Euclidean 空间 $\R^m$ 和 $\C^m$。

在第 4 和第 5 节中,我们回到最简单的情形,也就是在 $\R$ 中的收敛问题。借助实数的序结构,尤其是 $\R$ 的序完备性,我们推导出第一批 ``具体的收敛判别准则''。这些准则使我们能够求出许多重要数列的极限。除此之外,我们还将从 $\R$ 的序完备性出发,得到一个根本性的存在性原理——Bolzano-Weierstrass 定理。

第 6 节专门讨论度量空间中的 ``完备性'' 概念。将这一概念限制到赋范向量空间,就得到了 ``Banach 空间'' 的定义。这类空间最基本的例子就是 $\K^m$,但我们还将说明:有界函数构成的函数空间也是 Banach 空间。

Banach 空间在分析中无处不在,因此在本书的叙述中居于核心地位。即便如此,它们的结构仍然足够简单,使得初学者可以在不太费力的情况下,从理解实数自然过渡到理解 Banach 空间。再者,对这些空间的较早引入,也使得我们在后面章节中能够给出简洁而优美的证明。

为了内容的完整性,也为了读者整体(数学)素养的培养,我们在第 6 节中给出 Cantor 关于 ``序完备的有序域存在性'' 的证明,这个证明是通过对 $\Q$ 进行 ``完备化'' 而得到的。

在本章余下的各节中,我们将讨论级数的收敛性。在第 7 节里,我们会学习级数的基本性质,并讨论若干最重要的例子。这样一来,我们就能够研究实数的十进制表示以及其他形式的表示,从而证明实数集是不可数集。

在所有收敛级数当中,绝对收敛的级数具有尤为重要的地位。绝对收敛往往比较容易判别,这类级数在运算与处理上也相对简单。此外,在实际应用中许多重要的级数都是绝对收敛的。对于本章最后一节将要引入和研究的幂级数,这一点尤为明显。其中最重要的例子是指数级数,它的重要性将在后续章节中逐渐显现出来。

\section{收敛序列}

\subsection*{序列}