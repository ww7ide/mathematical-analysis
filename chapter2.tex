\chapter{收敛}

从本章开始, 我们终于进入分析的领域. 数学的这一分支在很大程度上是建立在 ``收敛'' 这一概念之上的. 借助收敛, 我们在某种意义上得以把无穷多个数 (或向量) ``加'' 在一起. 能够处理这样的无限运算, 正是分析与代数之间本质区别之所在.

试图把关于数列收敛的朴素想法加以公理化, 会自然地引出 ``距离''、``点的邻域'' 以及 ``度量空间'' 等概念——这正是第 1 节的主题. 在数列这一特殊情形中, 我们可以利用数域 $\K$ 的向量空间结构, 对这种情形下各种证明的分析表明: 只要有某种类似 ``绝对值'' 的概念可用, 其中大多数证明都可以推广到向量空间中的向量序列. 于是我们很自然地被引向去定义 ``赋范向量空间''——度量空间中一个尤其重要的类别.

在所有赋范向量空间中, 内积空间由于其结构更为丰富而显得格外重要, 并且它们的几何性质与我们熟悉的平面 Euclidean 几何非常相似. 事实上, 在初等分析中, 最重要的一类内积空间就是 $m$-维 Euclidean 空间 $\R^m$ 和 $\C^m$.

在第 4 和第 5 节中, 我们回到最简单的情形, 也就是在 $\R$ 中的收敛问题. 借助实数的序结构, 尤其是 $\R$ 的序完备性, 我们推导出第一批 ``具体的收敛判别准则''. 这些准则使我们能够求出许多重要数列的极限. 除此之外, 我们还将从 $\R$ 的序完备性出发, 得到一个根本性的存在性原理——Bolzano-Weierstrass 定理.

第 6 节专门讨论度量空间中的 ``完备性'' 概念. 将这一概念限制到赋范向量空间, 就得到了 ``Banach 空间'' 的定义. 这类空间最基本的例子就是 $\K^m$, 但我们还将说明: 有界函数构成的函数空间也是 Banach 空间.

Banach 空间在分析中无处不在, 因此在本书的叙述中居于核心地位. 即便如此, 它们的结构仍然足够简单, 使得初学者可以在不太费力的情况下, 从理解实数自然过渡到理解 Banach 空间. 再者, 对这些空间的较早引入, 也使得我们在后面章节中能够给出简洁而优美的证明.

为了内容的完整性, 也为了读者整体 (数学) 素养的培养, 我们在第 6 节中给出 Cantor 关于 ``序完备的有序域存在性'' 的证明, 这个证明是通过对 $\Q$ 进行 ``完备化'' 而得到的.

在本章余下的各节中, 我们将讨论级数的收敛性. 在第 7 节里, 我们会学习级数的基本性质, 并讨论若干最重要的例子. 这样一来, 我们就能够研究实数的十进制表示以及其他形式的表示, 从而证明实数集是不可数集.

在所有收敛级数当中, 绝对收敛的级数具有尤为重要的地位. 绝对收敛往往比较容易判别, 这类级数在运算与处理上也相对简单. 此外, 在实际应用中许多重要的级数都是绝对收敛的. 对于本章最后一节将要引入和研究的幂级数, 这一点尤为明显. 其中最重要的例子是指数级数, 它的重要性将在后续章节中逐渐显现出来.

\section{收敛序列}

在本节中, 我们考虑定义在自然数集上的函数, 因此这类函数只会取到可数多个值. 对这样的函数 $\varphi : \N \to X$, 我们特别关心的是它在 ``当 $n$ 趋于无穷大时'' 各个取值 $\varphi(n)$ 的行为. 由于我们实际上只能对 $\varphi$ 进行有限次求值, 也就是说, 我们永远不可能真正 ``到达无穷大'', 因此必须发展出一些方法, 使我们能够对 ``靠近无穷远处'' 的无穷多个函数值建立命题并加以证明. 这样的方法就构成了收敛序列的理论, 本节将对其进行介绍.

\subsection*{序列}

\begin{definition}[序列]
    设 $X$ 是集合. 称函数
    \[
    \varphi \colon \N \to X
    \]
    是 $X$ 中的序列, 记作
    \[
    (x_n) \;,\quad (x_n)_{n \in \N} \quad\text{或}\quad (x_0, x_1, x_2, \dots) \;,
    \]
    并称
    \[
    x_n \coloneqq \varphi(n)
    \]
    是序列的第 $n$ 项.
\end{definition}

\begin{definition}[数列]
    $\K$ 中的序列称为数列, 并将 $\K$-向量空间中的所有数列记作 $s$ 或 $s(\K)$. 进一步, $\R$ 中的序列称为实数列, $\C$ 中的序列称为复数列.
\end{definition}

\begin{remark}
    \begin{enumerate}
        \item 区分序列 $(x_n)$ 和它的像 $\set{x_n}{n \in \N}$ 是很重要的. 例如, 若对于任意 $n$ 有 $x_n = x$, 也就是说 $(x_n)$ 是一个常数列, 此时 $(x_n) = (x, x, x, \dots)$, 而 $\set{x_n}{n \in \N}$ 是单点集 $\set{x}$.
        \item 设 $(x_n)$ 是 $X$ 中的序列, $E$ 是某个性质. 称 $E$ 对 $(x_n)$ 的几乎所有项都成立, 若存在 $m \in \N$ 使得对所有 $n \geq m$ 命题 $E(x_n)$ 都为真. 也就是说, 除了至多有限多个 $x_n$ 以外, $E$ 对其余所有 $x_n$ 都成立. 当然, $E$ 也可以对若干个 (甚至全部) $n < m$ 的项成立. 若存在 $N \subseteq \N$ 满足 $|N| = \aleph_0$ 且 $E(x_n)$ 对所有 $n \in N$ 都成立, 则称 $E$ 对 $(x_n)$ 的无穷多项都成立. 例如, 实数列
        \[
        \Bigl(-5, 4, -3, 2, -1, 0, -\frac{1}{2}, \frac{1}{3}, -\frac{1}{4}, \frac{1}{5}, \dots, -\frac{1}{2n}, \frac{1}{2n+1}, \dots\Bigr)
        \]
        有无穷多个正项, 无穷多个负项, 并且对于几乎所有项都有其绝对值小于 $1$.
        \item 对于任意 $m \in \N^\ast$, 函数
        \[
        \psi \colon m + \N \to X
        \]
        也称为 $X$ 中的序列。也就是说,
        \[
        (x_j)_{j \geq m} = (x_m, x_{m+1}, x_{m+2}, \dots)
        \]
        是 $X$ 中的序列, 尽管其下标不是从 $0$ 开始. 这种约定是合理的, 因为通过映射
        \[
        \N \to m + \N \;,\quad n \mapsto m + n
        \]
        对下标进行 ``重新编号'' 之后, 这个 ``平移后的序列'' $(x_j)_{j \geq m}$ 就可以与通常意义下的序列 $(x_{m+k})_{k \in N} \in X^{\N}$ 对应起来.
    \end{enumerate}
\end{remark}

如果把复数列 $(z_n)_{n \geq 1}$ 的前几项画在复平面上, 其中 $z_n \coloneqq (1 - 1/n)(1 + \ii)$, 就会发现: 当 $n$ 增大时, 这些点 $z_n$ 会 ``任意接近'' $z \coloneqq 1 + \ii$. 换句话说, 随着 $n$ 的增大, $z_n$ 到 $z$ 的距离会变得 ``任意小''. 本节的目标, 就是把我们对于这类数列收敛的直观几何想法加以公理化, 使之能够推广应用到向量空间中的序列, 以及更抽象的集合中的序列上.

\begin{center}
\begin{tikzpicture}
    % 轴
    \draw[->] (-0.5,0) -- (4,0) node[right]{$\R$};
    \draw[->] (0,-0.5) -- (0,4) node[above]{$\ii \R$};
    % z1 在原点
    \fill (0,0) circle (1.5pt) node[below right]{$z_1$};
    % 序列点 z2, z3, z4, ...
    \fill (1.75,1.75) circle (1.5pt) node[below right]{$z_2$};
    \fill (2.1,2.1) circle (1.5pt) node[below right]{$z_3$};
    \fill (2.4,2.4) circle (1.5pt) node[below right]{$z_4$};
    % 中间的若干点(不标记)
    \foreach \x/\y in {2.6/2.6,2.7/2.7,3.2/3.2,3.3/3.3} \fill (\x,\y) circle (1pt);
    % 极限点 z
    \fill (3.5,3.5) circle (1.5pt) node[below right]{$z$};
\end{tikzpicture}
\end{center}

首先我们要意识到, ``距离'' 这一概念居于核心地位. 在数域 $\K$ 中, 我们可以借助绝对值函数来确定两点之间的距离. 若要研究某个任意集合 $X$ 中序列的收敛性, 我们首先需要在 $X$ 上赋予一种结构, 使得可以在 $X$ 的任意两个元素之间定义 ``距离''.

\subsection*{度量空间}

\begin{axiom}[度量]
    设 $X$ 是集合, 其上定义了函数
    \[
    d \colon X \times X \to \R \;.
    \]
    若对于任意 $x,y,z \in X$ 都有
    \begin{enumerate}
        \item\label{axiom:metric-1} $d(x,y) = 0$ 当且仅当 $x = y$.
        \item\label{axiom:metric-2} $d(x,y) = d(y,x)$.
        \item\label{axiom:metric-3} $d(x,y) \leq d(x,z) + d(z,y)$.
    \end{enumerate}
    则称 $d$ 是 $X$ 上的度量. 进一步, 称 $(X,d)$ 是度量空间. 最后, 称 $d(x,y)$ 是度量空间 $X$ 中点 $x$ 和 $y$ 之间的距离.
\end{axiom}

公理 \ref{axiom:metric-1}-\ref{axiom:metric-3} 显然是对 ``距离函数'' 非常自然的要求. 举例来说, 公理 \ref{axiom:metric-3} 可以被看作这样一条规则的公理化表述: ``从 $x$ 到 $y$ 的 `直达路径' 比先从 $x$ 到 $z$ 再从 $z$ 到 $y$ 的路径更短''.

\begin{definition}[开球和闭球]
    设 $(X,d)$ 是度量空间, $a \in X$, $r > 0$. 称集合
    \[
    \oball(a,r) \coloneqq \set{x \in X}{d(x,a) < r}
    \]
    是以 $a$ 为中心, $r$ 为半径的开球. 并称集合
    \[
    \cball(a,r) \coloneqq \set{x \in X}{d(x,a) \leq r}
    \]
    是以 $a$ 为中心, $r$ 为半径的闭球.
\end{definition}

\begin{example}
    \begin{enumerate}
        \item\label{ex:k-natural-metric} 函数
        \[
        \abs{\,\cdot\,} \colon \K \times \K \to \R \;,\quad (x,y) \mapsto \abs{x - y}
        \]
        是 $\K$ 上的度量, 称为自然度量. 除非另有说明, 否则默认 $\K$ 配备该度量, 并将其视为一个度量空间.
        \item 设 $(X,d)$ 是度量空间, $Y$ 是 $X$ 的非空子集. 则把 $d$ 限制在 $Y \times Y$ 上所得的函数
        \[
        d_Y \coloneqq d\vert_{Y \times Y}
        \]
        是 $Y$ 上的度量, 称为诱导度量. 进一步, 称 $(Y, d_Y)$ 是 $(X,d)$ 的子空间. 在不会引起混淆的情况下, 我们往往直接写 $d$, 而不特别区分 $d_Y$.
        \item $\C$ 的任意非空子集, 在从 $\C$ 的自然度量诱导出来的度量之下, 都是度量空间. 以这种方式在 $\R$ 上得到的度量, 正好就是在 \ref{ex:k-natural-metric} 中定义的那个自然度量.
        \item 设 $X$ 是非空集合. 则函数
        \[
        d \colon X \times X \to \R \;,\quad (x,y) \mapsto
        \begin{cases}
            1 \;, &x \neq y \;,\\
            0 \;, &x = y \;,
        \end{cases}
        \]
        是 $X$ 上的度量, 称为离散度量.
        \item 设 $(X_j, d_j),\, 1 \leq j \leq m$ 是度量空间, $X \coloneqq X_1 \times \cdots \times X_m$. 则函数
        \[
        d(x, y) \coloneqq \max_{1 \leq j \leq m} d_j(x_j, y_j)
        \]
        其中 $x \coloneqq (x_1, \dots, x_m) \in X$, $y \coloneqq (y_1, \dots, y_m) \in X$ 是 $X$ 上的度量, 称为积度量. 进一步, 称 $(X,d)$ 是 $(X_j, d_j),\, 1 \leq j \leq m$ 的积空间. 可以验证, 对于任意 $a \coloneqq (a_1, \dots, a_m) \in X$ 以及 $r > 0$, 都有
        \[
        \oball_X(a,r) = \prod_{j=1}^m \oball_{X_j}(a_j, r) \;,\quad \cball_X(a,r) = \prod_{j=1}^m \cball_{X_j}(a_j, r) \;.
        \]
    \end{enumerate}
\end{example}

度量公理的一个重要结论是反三角不等式 (见推论 \ref{cor:reversed-triangle-inequality}).

\begin{proposition}[反三角不等式]
    设 $(X,d)$ 是度量空间. 则
    \[
    d(x,y) \geq \abs{d(x,z) - d(z,y)} \;,\quad x,y,z \in X \;.
    \]
\end{proposition}

\begin{definition}[邻域]
    
\end{definition}