\usepackage{geometry}
\usepackage{amsthm}
\usepackage{amsmath,mathtools}
\usepackage{amssymb}
\usepackage{mathrsfs}
\usepackage{enumitem}
\usepackage{hyperref}
\usepackage{tikz}
\usepackage{pgfplots}
\usepackage{float}
\usepackage{multirow}

\ctexset{
    chapter = {
        name   = {},
        number = {第\arabic{chapter}章}
    }
}
\setcounter{secnumdepth}{1}

\theoremstyle{plain}
\newtheorem{definition}{定义}[section]
\newtheorem{axiom}[definition]{公理}
\newtheorem{proposition}{命题}[section]
\newtheorem{theorem}[proposition]{定理}
\newtheorem{corollary}[proposition]{推论}
\theoremstyle{definition}
\newtheorem{remark}{注}[section]
\newtheorem{example}{例}[section]
\newtheorem{exercises}{}[section]
\renewcommand{\theexercises}{\arabic{exercises}}

\makeatletter
\renewenvironment{cases}{
    \left\{\begin{matrix}
}{
    \end{matrix}\right.
}

\renewenvironment{proof}[1][\proofname]{
    \par
    \pushQED{\qed}
    \normalfont
    \topsep6\p@\@plus6\p@\relax
    \trivlist
    \item[\hskip\labelsep{\bfseries#1}\@addpunct{.}]
    \ignorespaces
}{
    \popQED\endtrivlist\@endpefalse
}

\newcommand{\set}{
    \@ifstar\set@star\set@nostar
}
% --------- 无 * ---------
\newcommand{\set@nostar}{
    \@ifnextchar[\set@nostar@opt\set@nostar@noopt
}
% --------- 无 * 无 [] ---------
\newcommand{\set@nostar@noopt}[1]{
    \@ifnextchar\bgroup{\set@nostar@noopt@two{#1}}{\set@nostar@noopt@one{#1}}
}
\newcommand{\set@nostar@noopt@one}[1]{
    \{\,#1\,\}
}
\newcommand{\set@nostar@noopt@two}[2]{
    \{\,#1 \;;\; #2\,\}
}
% --------- 无 * 有 [] ---------
\def\set@nostar@opt[#1]#2{
    \set@LR{#1}
    \@ifnextchar\bgroup{\set@nostar@opt@two{#2}}{\set@nostar@opt@one{#2}}
}
\def\set@LR#1{
    \def\set@L{\{}
    \def\set@R{\}}
    \ifx#1\big
        \def\set@L{\bigl\{}
        \def\set@R{\bigr\}}
    \else\ifx#1\Big
        \def\set@L{\Bigl\{}
        \def\set@R{\Bigr\}}
    \else\ifx#1\bigg
        \def\set@L{\biggl\{}
        \def\set@R{\biggr\}}
    \else\ifx#1\Bigg
        \def\set@L{\Biggl\{}
        \def\set@R{\Biggr\}}
    \fi\fi\fi\fi
}
\def\set@nostar@opt@one#1{
    \set@L\,#1\,\set@R
}
\def\set@nostar@opt@two#1#2{
    \set@L\,#1 \;;\; #2\,\set@R
}
% --------- 有 * ---------
\newcommand{\set@star}[1]{
    \@ifnextchar\bgroup{\set@star@two{#1}}{\set@star@one{#1}}
}
\newcommand{\set@star@one}[1]{
    \left\{\,#1\,\right\}
}
\newcommand{\set@star@two}[2]{
    \left\{\,#1 \;;\; #2\,\right\}
}
\makeatother

\DeclarePairedDelimiterX{\abs}[1]{\lvert}{\rvert}{
    #1
}

\newcommand{\power}{\mathcal P} % 幂集
\newcommand{\e}{\mathrm e} % 自然常数
\newcommand{\ii}{\mathrm i} % 虚数单位
\newcommand{\K}{\mathbb K} % 数域
\newcommand{\N}{\mathbb N} % 自然数集
\newcommand{\Q}{\mathbb Q} % 有理数集
\newcommand{\R}{\mathbb R} % 实数集
\newcommand{\C}{\mathbb C} % 复数集
\newcommand{\E}{\mathbb E} % 期望
\newcommand{\PP}{\mathbb P} % 概率
\newcommand{\topo}{\mathcal T} % 拓扑
\newcommand{\nbr}{\mathcal N} % 邻域族
\newcommand{\onbr}{\mathcal O} % 开邻域族
\newcommand{\cnbr}{\mathcal F} % 闭邻域族
\newcommand{\basis}{\mathcal B} % 拓扑基
\newcommand{\subbasis}{\mathcal S} % 拓扑子基
\newcommand{\ball}{\mathbb B} % 开球
\newcommand{\borel}{\mathcal B} % Borel sigma-代数
\newcommand{\lbg}{\mathcal L} % Lebesgue sigma-代数
\newcommand{\dd}{\mathop{}\!\mathrm d} % 微分算子
\newcommand{\mat}{\bm} % 矩阵和向量
\newcommand{\pto}{\xlongrightarrow{p}} % 依概率收敛
\newcommand{\dto}{\xlongrightarrow{d}} % 依分布收敛

\renewcommand{\implies}{\Longrightarrow} % 蕴含
\renewcommand{\impliedby}{\Longleftarrow} % 被蕴含
\renewcommand{\iff}{\Longleftrightarrow} % 等价

\DeclareMathOperator{\im}{\mathrm{im}} % 像
\DeclareMathOperator{\range}{\mathrm{ran}} % 值域
\DeclareMathOperator{\id}{\mathrm{id}} % 恒同映射
\DeclareMathOperator{\diam}{\mathrm{diam}} % 集合的直径
\DeclareMathOperator{\vol}{\mathrm{vol}} % 体积
\DeclareMathOperator{\supp}{\mathrm{supp}} % 支撑集
\DeclareMathOperator{\nullity}{\mathrm{null}} % 零化度
\DeclareMathOperator{\rank}{\mathrm{rk}} % 秩
\DeclareMathOperator{\trace}{\mathrm{tr}} % 迹
\DeclareMathOperator{\adj}{\mathrm{adj}} % 伴随
\DeclareMathOperator{\spec}{\mathrm{spec}} % 谱
\DeclareMathOperator{\var}{\mathrm{var}} % 方差
\DeclareMathOperator{\cov}{\mathrm{cov}} % 协方差
\DeclareMathOperator*{\overlim}{\overline{\lim}} % 上极限
\DeclareMathOperator*{\underlim}{\underline{\lim}} % 下极限

\let\oldRe\Re
\let\Re\relax
\DeclareMathOperator{\Re}{\oldRe} % 实部
\DeclareMathOperator{\Span}{\mathrm{span}} % 张成

%====================================
% ElegantBook 风格封面,重写 maketitle
%====================================
% 需要的宏包
\usepackage{hyperref}
\usepackage{geometry}
\usepackage{tikz}
\usetikzlibrary{calc}

% 颜色
\definecolor{structurecolor}{RGB}{60,113,183}
\definecolor{main}{RGB}{0,166,82}
\definecolor{second}{RGB}{255,134,24}
\definecolor{third}{RGB}{0,174,247}
\colorlet{coverlinecolor}{second}
\definecolor{darkgray}{gray}{0.30}

% 中文字体
\ifcsname heiti\endcsname
    \newcommand{\cbfseries}{\heiti}
\else
    \newcommand{\cbfseries}{\bfseries}
\fi
\ifcsname kaishu\endcsname
    \newcommand{\citshape}{\kaishu}
    \newcommand{\cnormal}{\kaishu}
\else
    \newcommand{\citshape}{\itshape}
    \newcommand{\cnormal}{\normalfont}
\fi

% 标签文字
\newcommand{\authorname}{\citshape 作者:}
\newcommand{\institutename}{\citshape 组织:}
\newcommand{\datename}{\citshape 时间:}
\newcommand{\versionname}{\citshape 版本:}

\makeatletter

% 对外接口命令
\newcommand{\subtitle}[1]{\gdef\@subtitle{#1}}
\newcommand{\institute}[1]{\gdef\@institute{#1}}
\newcommand{\logo}[1]{\gdef\@logo{#1}}
\newcommand{\cover}[1]{\gdef\@cover{#1}}
\newcommand{\extrainfo}[1]{\gdef\@extrainfo{#1}}
\newcommand{\version}[1]{\gdef\@version{#1}}
\newcommand\bioinfo[2]{\gdef\@bioinfo{{\citshape #1}:#2}}

% 重写 maketitle
\renewcommand{\maketitle}{
    \hypersetup{pageanchor=false}
    % \pagenumbering{Alph}
    \begin{titlepage}
        \newgeometry{margin=0in}
        \parindent=0pt

        % 顶部封面图
        \ifcsname @cover\endcsname
            \includegraphics[width=\linewidth]{\@cover}
        \else
            \@empty
        \fi

        % 彩色横条
        % \setlength{\fboxsep}{0pt}
        % \colorbox{coverlinecolor}{\makebox[\linewidth][c]{\shortstack[c]{\vspace{0.5in}}}}

        % 标题
        \vfill
        \vskip 15ex % \vskip-2ex
        \hspace{2em}
        \parbox{0.8\textwidth}{
            \bfseries\Huge
            \ifcsname @title\endcsname \cnormal\@title \fi
            \par
        }

        % 副标题 + 信息
        \vfill
        % \vspace{-1.0cm}
        \hspace{2.5em}
        \begin{minipage}[c]{0.67\linewidth}
            \color{darkgray}
            {\bfseries\Large
            \ifcsname @subtitle\endcsname \cnormal\@subtitle\\[2ex] \fi}
            \color{gray}
            % \normalsize
            {\renewcommand{\arraystretch}{1.2}
            \begin{tabular}{l}
                % \ifcsname @author\endcsname \authorname \@author\\ \fi
                \ifx\@author\empty\else \authorname\cnormal\@author\\ \fi
                \ifcsname @institute\endcsname \institutename\cnormal\@institute\\ \fi
                % \ifcsname @date\endcsname \@date\\ \fi
                \ifx\@date\empty\else \datename\cnormal\@date\\ \fi
                \ifcsname @version\endcsname \versionname\cnormal\@version\\ \fi
                \ifcsname @bioinfo\endcsname \cnormal\@bioinfo\\ \fi
            \end{tabular}}
        \end{minipage}

        % 右下角 logo
        \begin{minipage}[c]{0.27\linewidth}
            \begin{tikzpicture}[remember picture,overlay]
                \node[opacity=0.8,
                    anchor=south east,
                    outer sep=0pt,
                    inner sep=0pt] at
                    ($(current page.south east)+(-0.8in,1.5in)$)
                    {\ifcsname @logo\endcsname \includegraphics[width=4.2cm]{\@logo} \fi};
            \end{tikzpicture}
        \end{minipage}

        % 底部说明文字
        \vfill
        \begin{center}
            \parbox[t]{0.7\textwidth}{\centering\citshape
                \ifcsname @extrainfo\endcsname \@extrainfo \fi
            }
        \end{center}
        \vfill
    \end{titlepage}
    \restoregeometry
    \thispagestyle{empty}
}

\makeatother