\chapter{预备知识}

\section{逻辑基础}

\newpage

\section{集合}

\subsection{基本概念}

\subsection{幂集}

\subsection{集合的运算}

\subsection{Cartesian 积}

\subsection{集族}

\newpage

\section{函数}

\subsection{简单例子}

\subsection{函数的复合}

\subsection{交换图表}

\subsection{单射、满射和双射}

\subsection{反函数}

\subsection{集值函数}

\newpage

\section{关系与运算}

\subsection{等价关系}

\subsection{序关系}

\subsection{运算}

\begin{proposition}\label{prop:identity-element-is-unique}
    对于给定的运算, 至多存在一个单位元.
\end{proposition}

\begin{example}\label{eg:operations-on-function-spaces}
    设 $\circledast$ 是集合 $Y$ 上的运算, $X$ 是非空集合. 则我们通过 $\circledast$ 在 $\Funct(X,Y)$ 上诱导运算
    $$
    (f \odot g)(x) \coloneqq f(x) \circledast g(x)\ ,\quad x \in X\ .
    $$
    显然, 当 $\circledast$ 具有结合性和交换性时, $\odot$ 也具有. 若 $Y$ 关于 $\circledast$ 具有单位元 $e$, 则常值函数
    $$
    X \to Y\ ,\quad x \mapsto e
    $$
    是 $\Funct(X,Y)$ 关于 $\odot$ 的单位元.
\end{example}

\newpage

\section{自然数}

\subsection{Peano 公理}

\subsection{自然数的计算}

\begin{theorem}\label{thm:the-arithmetic-of-natural-numbers}
    
\end{theorem}

\subsection{除法}

\subsection{数学归纳法}

\begin{proposition}[良序原理]\label{prop:well-ordering-principle}
    自然数集 $\N$ 是良序的, 即 $\N$ 的每个非空子集都包含一个最小元.
\end{proposition}

\subsection{递归定义}

\begin{proposition}
    
\end{proposition}

\begin{example}
    
\end{example}

\begin{remark}
    
\end{remark}

\begin{example}\label{eg:recursive-definition-for-power-and-factorial}
    
\end{example}

\newpage

\section{可数性}

\subsection{排列}

\subsection{集合的等势}

\subsection{可数集}

\subsection{无穷 Cartesian 积}

\newpage

\section{群与同态}

在定理 \ref{thm:the-arithmetic-of-natural-numbers} 中, 我们在 $m \leq n$ 的前提下, 定义了两个自然数 $m$ 和 $n$ 的差 $n-m$. 同时, 在 $m$ 是 $n$ 约数的前提下, 我们也定义了两个自然数 $m$ 和 $n$ 的商 $n/m$. 在这两种情形中, 对 $m$ 和 $n$ 加上的这些限制正是为了保证所得的差和商仍然是自然数. 如果我们希望对任意自然数 $m$ 和 $n$ 定义 ``差'' $n-m$ 或 ``商'' $n/m$, 那么就必须离开自然数这一范围. 在第 9--11 节中, 我们将构造新类型的数, 把自然数集扩展到更大的数系之中, 在这些数系里, 这些运算可以 (几乎) 不受限制地使用.

当然, 这些新的数系必须被构造得仍然满足加法与乘法的通常法则. 为此, 脱离对某个具体数系的依赖, 而对这些运算法则本身进行研究是极其有用的. 这样的研究也为我们提供了进一步的训练: 如何从给定的定义和公理中, 以逻辑方式推演出命题.

对这里以及接下来几节中所出现问题的深入讨论, 严格来说属于代数学的范畴而非分析学, 因此我们在这里的叙述会相对简略, 只证明其中少数最重要的定理. 我们的目标是: 能够识别那些一次又一次以不同形式出现的一般代数结构. 从少量公理出发推导出大量算术运算法则, 使我们得以在本来杂乱庞大的公式和结论中建立起某种秩序, 并将注意力集中在真正本质的内容上. 从这些公理推出的命题, 只要公理成立, 那么它们就成立, 而与所处的具体背景无关. 已经证明过一次的结论, 就不必在每一个特殊情形中重新证明.

在本节以及接下来的各节中, 我们只给出少数几个具体的例子. 我们的主要目的是提供一套语言, 并希望读者在后面的章节中能够体会到这套语言的用处, 同时也能看到这套形式主义背后真正的数学内容.

\subsection{群}

\begin{axiom}[群]
    设 $G$ 是非空集合, 其上定义了运算
    $$
    \odot \colon G \times G \to G\ ,\quad (g,h) \mapsto g \odot h\ .
    $$
    若
    \begin{enumerate}[label={$(\mathrm G_{\arabic*})$}]
        \item $\odot$ 具有结合律.
        \item $\odot$ 具有单位元 $e$.
        \item 每个 $g \in G$ 都有逆元 $h \in G$ 使得 $g \odot h = h \odot g = e$.
    \end{enumerate}
    则称 $(G, \odot)$ 是群.
\end{axiom}

\begin{definition}[Abel 群]
    称群 $(G, \odot)$ 是交换群或 Abel 群, 若 $\odot$ 具有交换律.
\end{definition}

\begin{remark}\label{rem:group's-property}
    
\end{remark}

\subsection{子群}

\subsection{陪集}

\subsection{同态}

\begin{remark}\label{rem:group-homo}
    
\end{remark}

\subsection{同构}

\newpage

\section{环、域与多项式}

本节中, 我们考虑在其上定义了两种运算的集合. 在这里我们假定: 对于其中一种运算, 这个集合构成一个 Abel 群, 并且这两种运算满足某种合适的 ``分配律''. 由此发展出 ``环'' 和 ``域'' 的概念, 它们把运算法则加以形式化. 作为环的特别重要的例子, 我们将考察幂级数环以及单变量 (和多变量) 多项式环, 并推导它们的一些基本性质. 多项式函数在运算上相对容易处理, 在分析中也具有重要地位, 因为 ``复杂函数可以被多项式任意逼近'', 这一说法我们将在后文中以更精确的形式加以表述.

\subsection{环}

\begin{axiom}[环]\label{ax:ring}
    设 $R$ 是非空集合, 其上定义了加法运算
    $$
    + \colon R \times R \to R\ ,\quad (a,b) \mapsto a + b\ ,
    $$
    和乘法运算
    $$
    \times \colon R \times R \to R\ ,\quad (a,b) \mapsto a \times b\ .
    $$
    若
    \begin{enumerate}[label={$(\mathrm R_{\arabic*})$}]
        \item $(R,+)$ 是 Abel 群.
        \item $\times$ 具有结合律.
        \item $\times$ 对 $+$ 具有分配律:
        $$
        (a + b) \times c = a \times c + b \times c\ ,\quad c \times (a + b) = c \times a + c \times b\ ,\quad a,b,c \in R\ .
        $$
    \end{enumerate}
    则称 $(R, +, \times)$ 是环.
\end{axiom}

在这里, 我们采用通常的约定, 即乘法优先于加法. 例如, $a \times b + c$ 的意思是 $(a \times b) + c$ (也就是说, 先进行乘法 $d \coloneqq a \times b$, 再计算加法 $d+c$), 而不是 $a \times (b+c)$. 另外, 在不会引起混淆的情况下, 我们通常将 $a \times b$ 简写为 $ab$.

\begin{definition}[交换环]
    称环 $(R, +, \cdot)$ 是交换环, 若 $\cdot$ 具有交换律. 在这种情况下, 分配律 \hyperref[ax:ring]{$(\mathrm R_3)$} 可以简化为
    $$
    (a + b)c = ac + bc\ ,\quad a,b,c \in R\ .
    $$
\end{definition}

\begin{definition}[含幺环]
    称环 $(R, +, \cdot)$ 是含幺环, 若 $\cdot$ 具有单位元. 并将该单位元记作 $1_R$ 或 $1$, 称为 $R$ 的幺元或乘法单位元.
\end{definition}

当加法和乘法运算在上下文中明确时, 我们把 $(R, +, \cdot)$ 简写为 $R$.

\begin{remark}\label{rem:arithmetic-rules-in-rings}
    设 $R \coloneqq (R, +, \cdot)$ 是环.
    \begin{enumerate}
        \item 环 $R$ 的加法群 $(R,+)$ 的单位元, 如例 \ref{eg:recursive-definition-for-power-and-factorial} 中一样, 记作 $0_R$ 或 $0$, 称为 $R$ 的零元或加法单位元. 根据命题 \ref{prop:identity-element-is-unique} 可知, 若 $0_R$ 或 $1_R$ 存在, 则唯一.
        \item 由注 \ref{rem:group's-property}.4 可知, 对于每个 $a \in R$ 都有 $-(-a) = a$.
        \item 对于每一对 $a,b \in R$, 由注 \ref{rem:group's-property}.3 可知, 方程 $a + x = b$ 在 $R$ 中有唯一的解 $x$, 即 $x = b - a \coloneqq b + (-a)$ (读作 $b$ 减 $a$), 称为 $a$ 与 $b$ 的差.
        \item 对于任意 $a \in R$, 有 $0a = a0 = 0$ 且 $-0 = 0$. 进一步, 令 $a \neq 0$, 若存在非零元 $b \in R$ 使得 $ab = 0$ 或 $ba = 0$, 则称 $a$ 是 $R$ 的零因子. 若 $R$ 是交换环且没有零因子, 即 $ab = 0$ 蕴含 $a = 0$ 或 $b = 0$, 则称 $R$ 是整环.
        \item 对于任意 $a,b \in R$, 有 $-ab \coloneqq -(ab) = a(-b) = (-a)b$ 且 $(-a)(-b) = ab$.
        \item 若 $R$ 是含幺环, 则对于任意 $a \in R$ 都有 $(-1)a = -a$.
        \item 由例 \ref{eg:recursive-definition-for-power-and-factorial} 可知, 对于任意 $n \in \N$ 和 $a \in R$, $n \cdot a = na$ 是良定义的且本例中的运算法则都成立. 特别的, $0_\N \cdot a \coloneqq 0_R$. 又由注 \ref{rem:arithmetic-rules-in-rings}.4 有 $0_R \cdot a = 0_R$, 因此把 $0_\N$ 和 $0_R$ 的下标省略也不会产生歧义. 类似的, 若 $R$ 是含幺环, 则 $1_\N \cdot a = 1_R \cdot a = a$.
    \end{enumerate}
\end{remark}

\begin{example}\label{eg:rings}
    \begin{enumerate}
        \item 平凡环只含有一个元素 $0$, 并且通常也记作 $0$. 含有超过一个元素的环称为非平凡环. 平凡环显然是含幺的交换环. 若 $R$ 是含幺环, 于是对于每个 $a \in R$ 都有 $1_R \cdot a = a$, 则 $R$ 是平凡的当且仅当 $1_R = 0_R$.
        \item 设 $R$ 是环, $X$ 是非空集合. 则 $R^X$ 在定义如下运算后是环.
        $$
        (f+g)(x) \coloneqq f(x) + g(x)\ ,\quad (f g)(x) \coloneqq f(x) g(x)\ ,\quad x \in X\ ,\quad f,g \in R^X\ .
        $$
        若 $R$ 是交换环或含幺环, 则 $R^X$ 也是 (见例 \ref{eg:operations-on-function-spaces}). 特别的, 当 $m \geq 2$ 时, 环 $R$ 的直积 $R^m$ 在定义如下运算后是环.
        $$
        a + b \coloneqq (a_1 + b_1, \ldots, a_m + b_m)\ ,\quad a b \coloneqq (a_1 b_1, \ldots, a_m b_m)\ ,\quad a,b \in R^m\ .
        $$
        若 $R$ 是非平凡的含幺环且 $X$ 有至少两个元素, 则 $R^X$ 有零因子.
        \item 设 $S$ 是环 $R$ 的非空子集. 若
        \begin{enumerate}[label={$(\mathrm{SR}_{\arabic*})$}]
            \item $S$ 是 $(R,+)$ 的子群.
            \item $S$ 对乘法封闭, 即 $S \cdot S \subseteq S$.
        \end{enumerate}
        则称 $S$ 是 $R$ 的子环. 显然, $0 = \{0\}$ 和 $R$ 都是 $R$ 的子环. 即便 $R$ 是含幺环, $S$ 却不一定含有幺元 (见例 \ref{eg:rings}.5). 不过, 若 $1_R \in S$, 则 $1_R$ 是 $S$ 的幺元. 当然, 若 $R$ 是交换环, 则 $S$ 也是, 但反过来一般不成立.
        \item 子环的交仍然是子环.
        \item 设 $R$ 是非平凡的含幺环,
        $$
        S \coloneqq \set{g \in R^\N}{\text{对于几乎所有}\ n \in \N\ \text{都有}\ g(n) = 0\ \text{(即除有限多个外)}}\ .
        $$
        则 $S$ 是 $R^\N$ 的子环且不含幺元.
        \item 设 $X$ 是集合, 在 $\power(X)$ 上定义对称差
        $$
        A \vartriangle B \coloneqq (A \cup B) \setminus (A \cap B) = (A \setminus B) \cup (B \setminus A)\ ,\quad A,B \in \power(X)\ .
        $$
        则 $\bigl(\power(X), \vartriangle, \cap\bigr)$ 是含幺的交换环.
    \end{enumerate}
\end{example}

\begin{definition}[环同态]
    设 $R$ 和 $R'$ 是环. 称函数 $\varphi \colon R \to R'$ 是环同态, 若 $\varphi$ 与环运算相容, 即
    $$
    \varphi(a+b) = \varphi(a) + \varphi(b)\ ,\quad \varphi(ab) = \varphi(a) \varphi(b)\ ,\quad a,b \in R\ .
    $$
    特别的, 若 $R = R'$, 则称 $\varphi$ 是环自同态.
\end{definition}

\begin{definition}[环同构]
    设 $\varphi \colon R \to R'$ 环同态. 若 $\varphi$ 是双射, 则称 $\varphi$ 是环同构, 并称 $R$ 和 $R'$ 同构. 特别的, 若 $R = R'$, 则称 $\varphi$ 是环自同构.
\end{definition}

\begin{remark}\label{rem:ring-homomorphism}
    \begin{enumerate}
        \item 环同态 $\varphi \colon R \to R'$, 特别的, 还是群同态 $\varphi \colon (R,+) \to (R',+)$. $\varphi$ 的核就是这个群同态的核, 即
        $$
        \ker \varphi \coloneqq \set{a \in R}{\varphi(a) = 0} = \varphi^{-1}(0)\ .
        $$
        \item 零函数
        $$
        \varphi \colon R \to R'\ ,\quad a \mapsto 0_{R'}
        $$
        是同态且 $\ker \varphi = R$.
        \item 设 $R$ 和 $R'$ 是含幺环, $\varphi \colon R \to R'$ 是同态. 正如注 \ref{rem:ring-homomorphism}.2 所示, $\varphi(1_R) = 1_{R'}$ 不一定成立. 这一问题可以看作是, 由于环对乘法这一运算并不构成群所导致的.
    \end{enumerate}
\end{remark}

\subsection{二项式定理}

接下来, 我们将证明环公理 \hyperref[ax:ring]{$(\mathrm R_1)$}--\hyperref[ax:ring]{$(\mathrm R_3)$} 除了得到注 \ref{rem:arithmetic-rules-in-rings} 的结论外还有其他重要的结果.

\begin{theorem}[二项式定理]
    设 $a$ 和 $b$ 是含幺环 $R$ 的两个可交换元 (即 $ab = ba$). 则
    $$
    (a + b)^n = \sum_{k = 0}^n \binom{n}{k} a^k b^{n - k}\ ,\quad n \in \N\ .
    $$
\end{theorem}

\subsection{多项式定理}

\subsection{域}

\begin{axiom}[域]
    设 $K$ 是非空集合, 其上定义了加法运算
    $$
    + \colon K \times K \to K\ ,\quad (a,b) \mapsto a + b\ ,
    $$
    和乘法运算
    $$
    \times \colon K \times K \to K\ ,\quad (a,b) \mapsto a \times b\ .
    $$
    若
    \begin{enumerate}[label={$(\mathrm F_{\arabic*})$}]
        \item $(K, +, \times)$ 是含幺的交换环.
        \item $(K \setminus \{0\}, \times)$ 是 Abel 群.
    \end{enumerate}
    则称 $(K, +, \times)$ 是域.
\end{axiom}

\subsection{有序域}

\subsection{形式幂级数}

\begin{definition}[形式幂级数]
    设 $R$ 是非平凡的含幺环. 在 $R^\N$ 上定义加法运算
    $$
    (p + q)_n \coloneqq p_n + q_n\ ,\quad n \in \N\ ,\quad p,q \in R^\N\ ,
    $$
    并通过卷积定义乘法运算
    $$
    (p \cdot q)_n \coloneqq \sum_{j=0}^n p_j q_{n-j} = p_0 q_n + p_1 q_{n-1} + \cdots + p_n q_0\ ,\quad n \in \N\ ,\quad p,q \in R^\N\ ,
    $$
    其中 $p_n \coloneqq p(n)$ 称为 $p$ 的第 $n$ 项系数. 在这种情况下, 称 $p \in R^\N$ 是 $R$ 上的形式幂级数, 并记 $R[\![X]\!] \coloneqq (R^\N, +, \cdot)$.
\end{definition}

下面的命题表明 $R[\![X]\!]$ 是环. 注意, 这个环与例 \ref{eg:rings}.2 中介绍的函数环 $R^\N$ 并不相同.

\begin{proposition}
    $R[\![X]\!]$ 是含幺环, 称为 $R$ 上的形式幂级数环. 若 $R$ 是交换环, 则 $R[\![X]\!]$ 也是.
\end{proposition}

\subsection{多项式}

\subsection{多项式函数}

\subsection{多项式的除法}

\subsection{线性因子}

\subsection{多变量多项式}

\subsection{练习}

\begin{exercises}\label{ex:difference-of-powers-formula}
    设 $a$ 和 $b$ 是含幺环的可交换元, $n \in \N$. 证明
    \begin{enumerate}
        \item $a^{n+1} - b^{n+1} = (a-b) \sum_{j=0}^n a^j b^{n-j}$.
        \item $a^{n+1} - 1 = (a-1) \sum_{j=0}^n a^j$.
    \end{enumerate}
\end{exercises}

\begin{exercises}\label{ex:inequality-on-ordered-field}
    设 $K$ 是有序域, $a,b,c,d \in K$.
    \begin{enumerate}
        \item 证明不等式 $\displaystyle \frac{|a+b|}{1 + |a+b|} \leq \frac{|a|}{1 + |a|} + \frac{|b|}{1 + |b|}$.
        \item 证明, 若 $b>0$, $d>0$ 且 $\displaystyle \frac{a}{b} < \frac{c}{d}$, 则 $\displaystyle \frac{a}{b} < \frac{a+c}{b+d} < \frac{c}{d}$.
        \item 证明, 若 $a,b \in K^\ast$, 则 $\displaystyle \abs[\Big]{\frac{a}{b} + \frac{b}{a}} \geq 2$.
    \end{enumerate}
\end{exercises}

\newpage

\section{有理数}

\subsection{整数}

\subsection{有理数}

\begin{proposition}\label{prop:Q-is-countable}
    $\Z$ 和 $\Q$ 是可数集.
\end{proposition}

\subsection{多项式的有理零点}

\subsection{平方根}

\newpage

\section{实数}

\subsection{序完备性}

\subsection{Dedekind 对实数的构造}

\subsection{$\R$ 上的自然序结构}

\subsection{广义实数}

\subsection{上确界与下确界的刻画}

\subsection{Archimedes 公理}

\begin{proposition}[Archimedes 公理]
    $\N$ 在 $\R$ 中没有上界, 即对于每个 $x \in \R$ 都存在 $n \in \N$ 使得 $n > x$.
\end{proposition}

\begin{corollary}\label{cor:archimedean-property-cor}
    \begin{enumerate}
        \item 设 $a \in \R$. 若对于任意 $n \in \N^\ast$ 都有 $0 \leq a \leq 1/n$, 则 $a = 0$.
        \item 对于每个 $a \in \R$ 且 $a > 0$ 都存在 $n \in \N^\ast$ 使得 $1/n < a$.
    \end{enumerate}
\end{corollary}

\subsection{有理数在 $\R$ 中的稠密性}

\begin{proposition}\label{prop:Q-density-in-R}
    对于任意 $a,b \in \R$ 使得 $a < b$, 都存在 $r \in \Q$ 使得 $a < r < b$.
\end{proposition}

\subsection{$n$ 次方根}

\subsection{无理数在 $\R$ 中的稠密性}

\subsection{区间}

\newpage

\section{复数}

\subsection{复数的构造}

\subsection{基本性质}

\subsection{复数的运算}

\begin{proposition}\label{prop:abs-in-Q}
    设 $z,w \in \C$.
    \begin{enumerate}
        \item $|zw| = |z| |w|$.
        \item 对于任意 $z \in \R$ 都有 $|z|_\C = |z|_\R$.
        \item $|\Re z| \leq |z|$, $|\Im z| \leq |z|$, $|z| = |\overline z|$.
        \item $|z| = 0$ 当且仅当 $z = 0$.
        \item (三角不等式) $|z + w| \leq |z| + |w|$.
        \item 对于任意 $z \in \C^\ast$ 都有 $z^{-1} = 1/z = \overline z / |z|^2$.
    \end{enumerate}
\end{proposition}

\begin{corollary}[反三角不等式]\label{cor:reversed-triangle-inequality-in-Q}
    $$
    |z - w| \geq \abs[\big]{|z| - |w|}\ ,\quad z,w \in \C\ .
    $$
\end{corollary}

\subsection{$\K$ 中的球}

\newpage

\section{向量空间、仿射空间与代数}

线性代数无疑是所有数学研究领域中最为肥沃的一支, 并且为数学各个分支中许多影响深远的理论奠定了基础. 尤其是, 线性代数是分析学的主要工具之一, 因此本节将介绍其基本概念, 并配以例子加以说明. 我们的目标依然是: 能够识别那些在后续章节中以不同形式频繁出现的简单代数结构. 若要作更深入的研究, 读者可参阅线性代数的丰富文献, 例如: [Art91]、[Gab96]、[Koe83]、[Wal82] 和 [Wal85].

在下文中, $K$ 表示一个任意的域.

\subsection{向量空间}

\begin{axiom}[向量空间]
    设 $(K, +, \times)$ 是域, $V$ 是非空集合, 其上定义了 ``内'' 运算向量加法
    $$
    \oplus \colon V \times V \to V\ ,\quad (v,w) \mapsto v \oplus w\ ,
    $$
    和 ``外'' 运算标量乘法
    $$
    \cdot \colon K \times V \to V\ ,\quad (\lambda, v) \mapsto \lambda \cdot v\ .
    $$
    若
    \begin{enumerate}[label={$(\mathrm{VS}_{\arabic*})$}]
        \item $(V, \oplus)$ 是 Abel 群.
    \end{enumerate}
    且对于任意 $\lambda, \mu \in K$ 和 $v,w \in V$ 都有
    \begin{enumerate}[label={$(\mathrm{VS}_{\arabic*})$},start=2]
        \item $1_K$ 是 $\cdot$ 的单位元:
        $$
        1_K \cdot v = v\ .
        $$
        \item $\cdot$ 对 $\oplus$ 具有分配律:
        $$
        \lambda \cdot (v \oplus w) = \lambda \cdot v \oplus \lambda \cdot w\ .
        $$
        \item $\cdot$ 对 $+$ 具有分配律:
        $$
        (\lambda + \mu) \cdot v = \lambda \cdot v + \mu \cdot v\ .
        $$
        \item $\cdot$ 和 $\times$ 相容:
        $$
        \lambda \cdot (\mu \cdot v) = (\lambda \times \mu) \cdot v\ .
        $$
    \end{enumerate}
    则称 $(V, \oplus, \cdot)$ 是域 $K$ 上的向量空间或 $K$-向量空间.
\end{axiom}

\subsection{线性映射}

\begin{definition}[线性映射]
    设 $V$ 和 $W$ 是 $K$-向量空间. 称函数 $T \colon V \to W$ 是 $K$-线性映射或 $K$-线性的, 若 $T$ 与向量空间运算相容, 即
    $$
    T(\lambda v + \mu w) = \lambda T v + \mu T w\ ,\quad \lambda, \mu \in K\ ,\quad v,w \in V\ .
    $$
\end{definition}

\begin{definition}[向量空间同态]
    设 $V$ 和 $W$ 是向量空间. 称线性映射 $T \colon V \to W$ 是向量空间同态, 并将从 $V$ 到 $W$ 的所有线性映射记作 $\Hom(V,W)$. 
    特别的, 若 $V = W$, 则称 $T$ 是向量空间自同态, 并将 $V$ 上的所有向量空间自同态记作 $\End(V) \coloneqq \Hom(V,V)$.
\end{definition}

\begin{definition}[向量空间同构]
    设 $T \colon V \to W$ 是向量空间同态. 若 $T$ 是双射, 则称 $T$ 是向量空间同构, 并称 $V$ 和 $W$ 同构, 记作 $V \cong W$. 特别的, 若 $V = W$, 则称 $T$ 是向量空间自同构.
\end{definition}

\begin{remark}
    \begin{enumerate}
        \item 对于线性映射 $T \colon V \to W$, 当 $v \in V$ 时, 在不会引起混淆的情况下, 通常把 $T(v)$ 记作 $T v$.
        \item 向量空间同态 $T \colon V \to W$, 特别的, 还是群同态 $T \colon (V,+) \to (W,+)$. 因此, $T0 = 0$ 且对于任意 $v \in V$ 都有 $T(-v) = -Tv$. $T$ 的核或零空间就是这个群同态的核, 即
        $$
        \ker T = \set{v \in V}{Tv = 0} = T^{-1} 0\ .
        $$
        因此, $T$ 是单射当且仅当其核是平凡的, 即 $\ker T = \{0\}$ (见注 \ref{rem:group-homo} 的 1 和 4).
    \end{enumerate}
\end{remark}

\begin{example}\label{eg:vector-spaces}
    设 $V$ 和 $W$ 是 $K$-向量空间.
    \begin{enumerate}
        \item 一个零或平凡 (向量) 空间只包含一个向量 $0$, 通常直接记作 $0$. 任何其他向量空间都是非平凡的.
        \item 设 $U$ 是 $V$ 的非空子集. 若
        \begin{enumerate}[label={$(\mathrm{SS}_{\arabic*})$}]
            \item $U$ 是 $(V,+)$ 的子群.
            \item $U$ 对标量乘法封闭, 即 $K \cdot U \subseteq U$.
        \end{enumerate}
        则称 $U$ 是 $V$ 的子空间. 容易验证, $U$ 是 $V$ 的子空间, 当且仅当 $U$ 对 $V$ 的两种运算都封闭, 即
        $$
        U + U \subseteq U\ ,\quad K \cdot U \subseteq U\ .
        $$
        \item 设 $T \colon V \to W$ 是线性映射. 则 $\ker T$ 是 $V$ 的子空间, $\im T$ 是 $W$ 的子空间. 若 $T$ 是单射, 则 $T^{-1} \in \Hom(\im T, V)$.
        \item 当把域上的运算视为向量空间上的运算时, $K$ 自身就是以 $K$ 为数域的向量空间.
        \item 设 $X$ 是集合. 则 $V^X$ 在定义如下运算 (见例 \ref{eg:operations-on-function-spaces}) 后是 $K$-向量空间.
        $$
        (f+g)(x) \coloneqq f(x) + g(x)\ ,\quad (\lambda f)(x) \coloneqq \lambda f(x)\ ,\quad x \in X\ ,\quad \lambda \in K\ ,\quad f,g \in V^X\ .
        $$
        特别的, $K^m\ (m \in \N^\ast)$ 在定义如下运算后是 $K$-向量空间.
        $$
        x + y \coloneqq (x_1 + y_1, \ldots, x_m + y_m)\ ,\quad \lambda x \coloneqq (\lambda x_1, \ldots, \lambda x_m)\ ,\quad \lambda \in K\ ,\quad x,y \in K^m\ .
        $$
        显然, $K^1$ 和 $K$ (作为 $K$-向量空间) 是相同的.
    \end{enumerate}
\end{example}

\subsection{向量空间的基}

\subsection{仿射空间}

\subsection{仿射映射}

\subsection{多项式插值}

\subsection{代数}

\begin{axiom}[代数]
    设 $(A, \oplus, \cdot)$ 是 $K$-向量空间, 其上定义了向量乘法
    $$
    \otimes \colon A \times A \to A\ ,\quad (a,b) \mapsto a \otimes b\ .
    $$
    若
    \begin{enumerate}[label={$(\mathrm A_{\arabic*})$}]
        \item $(A, \oplus, \otimes)$ 是环.
        \item $\otimes$ 和 $\cdot$ 相容:
        $$
        \lambda \cdot (v \otimes w) = (\lambda \cdot v) \otimes w = v \otimes (\lambda \cdot w)\ ,\quad \lambda \in K\ ,\quad v,w \in A\ .
        $$
    \end{enumerate}
    则称 $(A, \oplus, \cdot, \otimes)$ 是域 $K$ 上的代数或 $K$-代数.
\end{axiom}

\begin{example}\label{eg:algebras}
    \begin{enumerate}
        \item 设 $B$ 是 $K$-代数 $A$ 的非空子集. 若
        \begin{enumerate}[label={$(\mathrm{SA}_{\arabic*})$}]
            \item $B$ 是 $A$ 的子空间.
            \item $B$ 对向量乘法封闭, 即 $B \otimes B \subseteq B$.
        \end{enumerate}
        则称 $B$ 是 $A$ 的子代数.
        \item 设 $X$ 是非空集合. 则 $K^X$ 在定义例 \ref{eg:rings}.2 和例 \ref{eg:vector-spaces}.5 的运算后是含幺的交换 $K$-代数.
    \end{enumerate}
\end{example}

\subsection{差分算子与求和公式}

\subsection{Newton 插值多项式}